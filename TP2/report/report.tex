
%% Question 1.1

Hadoop system divides the tasks across many nodes, which can have different performances, due for example to hardware reasons.
Thus, one node can slow down the whole system.
To address this issue, a popular approach is speculative tasks.
Tasks which are still running on some nodes are launched again on nodes that have already finished their tasks, thinking that it may be faster.

In this question, we aim to observe this experimentally.
We generated with \texttt{randomwriter} a dataset of 20 Go.
We run sort benchmark, with and without speculation.
We then re-do the same, but we also stress all the CPUs of one node during the executions.

\todo{ figure + explain results }

%% Question 1.2

During an execution, a node failure may happen.
The JobTracker receives regular heartbeat signal from the TaskTracker nodes.
If it receives no heartbeat from a node, it waits for a given amount of time before considering it as dead.
This is the expiry time.
When it realize a node is dead, the JobTracker has to resubmit the dead node tasks to other nodes,
and also rerun some previous completed tasks.

In this question, we try to observe this experimentally.
We generated with \texttt{randomwriter} a dataset of 5 Go.
We run executions with two different expiry times : 30 seconds and 60 seconds.
During an execution, we kill the TaskTracker daemon on one node, with two different scenarios:
\begin{itemize}
    \item Before the completion of map tasks
    \item After the completion of map tasks
\end{itemize}

\todo{ figure + explain results }

%% Question 2.1

During an execution, it is possible to choose when to start the reduce tasks compared to the completed map tasks.
At first, it may seem better to wait for all map tasks to end before running reduce tasks,
but in some situations starting reduce tasks earlier may lead to better performances.
This optimization is application dependent.
Starting reduce tasks earlier spreads out the data transfer over time, which is good if performances are limited by the network.
But on the other hand, reduce tasks occupy slots that may have been used by other tasks.

In this question, we generated with \texttt{randomwriter} a dataset of 5 Go, and with \texttt{randomtextwriter} a dataset of 1 Go.
We denote the reduce start paramater by $p$.
This mean that reduce tasks start when fraction $p$ of map tasks are completed.
We set $p \in \lbrace 0.05, 0.5, 1 \rbrace$.
We run sort benchmark on the 5 Go dataset, and the word count benchmark on the 1 Go dataset.

\todo{ figure + explain results }