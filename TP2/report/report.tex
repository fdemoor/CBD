\documentclass[a4paper]{article}
\usepackage[francais]{babel}
\usepackage[utf8]{inputenc}
\usepackage[T1]{fontenc}
\usepackage[margin=2.5cm]{geometry}
\usepackage{graphicx}
\usepackage{xspace}
\usepackage{authblk}

\usepackage{xcolor}

\usepackage{hyperref}
\usepackage{diagbox}
\usepackage{listings}

\newcommand{\todo}[1]{\textcolor{red}{\textbf{ToDo} [ \emph{#1} ]}}
\newcommand{\fixme}[1]{\textcolor{purple}{\textbf{FixMe} [ \emph{#1} ]}}
\newcommand{\note}[1]{\textcolor{blue}{\textbf{Note} [ \emph{#1} ]}}

\lstdefinestyle{log}{
  emptylines=1,
  breaklines=true,
  moredelim=**[is][\color{blue}]{@}{.},
  moredelim=**[is][\color{purple}]{!}{.},
  moredelim=**[is][\color{red}]{?}{.},
  moredelim=**[is][\color{orange}]{£}{.},
  xleftmargin=\parindent,
  basicstyle=\scriptsize\sffamily
}


\definecolor{silver}{rgb}{0.7,0.7,0.7}
\definecolor{green}{rgb}{0,0.6,0}

\lstdefinestyle{C}{
  belowcaptionskip=1\baselineskip,
  breaklines=true,
  xleftmargin=\parindent,
  language=C,
  showstringspaces=false,
  basicstyle=\footnotesize\ttfamily,
  keywordstyle=\bfseries\color{green},
  commentstyle=\itshape\color{silver},
  identifierstyle=\color{black},
  stringstyle=\color{red},
  frame=l,
  numbers=left
}

\graphicspath{{logs/}}





\begin{document}

\title{CBD - Practical evaluation}
\author{Florestan de Moor}
\author{Rémi Hutin}
\affil{ENS Rennes, Université de Rennes 1}
\date{24 Mars 2017}

\maketitle


\section*{Introduction}

\todo{ ??? }

\section{Mitigating stragglers/failures in Hadoop}

\paragraph{Question 1.1}

Hadoop system divides the tasks across many nodes, which can have different performances, due for example to hardware reasons.
Thus, one node can slow down the whole system.
To address this issue, a popular approach is speculative tasks.
Tasks which are still running on some nodes are launched again on nodes that have already finished their tasks, thinking that it may be faster.

In this question, we aim to observe this experimentally.
We used 5 nodes in \texttt{paravance} cluster.
We generated with \texttt{randomwriter} a dataset of 20 Go.
We run sort benchmark, with and without speculation.
We then re-do the same, but we also stress all the CPUs of one node during the executions.

The results are available in appendix:

\begin{table}[!ht]
    \centering
\begin{tabular}{|c|c|c|}
    \hline
    \backslashbox{Speculation}{Stress} & Without & With \\
    \hline
                False             &   \figurename~\ref{1.1.specOff.noStress}   &  \figurename~\ref{1.1.specOff.Stress}   \\
    \hline
                True             &   \figurename~\ref{1.1.specOn.noStress}   &  \figurename~\ref{1.1.specOn.Stress}    \\
    \hline
\end{tabular}
\end{table}

The interesting figures are the one with stress.
When speculation is not enabled, we can see that the node stressed (\texttt{paravance-36}) is clearly slowing down the whole execution.
It takes indeed around 1.4 more time than the other nodes to complete its tasks (35 against 25 seconds).
With speculation enabled, we can see that we have killed tasks on this node:
speculative tasks were launched on other nodes and, as they finished earlier, the original tasks were killed.
This is more efficient: the stressed node is only around 1.16 slower than the other nodes (70 against 60 seconds).
Speculative tasks thus seem to improve the performances of the application, as expected.


\paragraph{Question 1.2}

During an execution, a node failure may happen.
The JobTracker receives regular heartbeat signal from the TaskTracker nodes.
If it receives no heartbeat from a node, it waits for a given amount of time before considering it as dead.
This is the expiry time.
When it realizes a node is dead, the JobTracker has to resubmit the dead node tasks to other nodes,
and also rerun some previous completed tasks.

In this question, we try to observe this experimentally.
We used 5 nodes in \texttt{parapluie} cluster.
We generated with \texttt{randomwriter} a dataset of 5 Go.
We run executions with two different expiry times : 30 seconds and 60 seconds.
During an execution, we kill the TaskTracker daemon on one node, with two different scenarios:
\begin{itemize}
    \item Before the completion of map tasks
    \item After the completion of map tasks
\end{itemize}

\begin{table}[!ht]
    \centering
\begin{tabular}{|c|c|c|}
    \hline
    \backslashbox{Expiry time}{Completion of map tasks} & Before & After \\
    \hline
                30 seconds             &   -   &  \figurename~\ref{1.2.30s.reduce}   \\
    \hline
                60 seconds             &   -   &  \figurename~\ref{1.2.60s.reduce}    \\
    \hline
\end{tabular}
\end{table}

We can see that there is not a significant difference between 30 and 60 seconds expiry time.
In the latter, the JobTracker takes more time to detect the dead nodes, and thus to re-launch tasks, resulting in a slower execution.
When a node TaskTracker is killed after the completion of map tasks,
we observe that reduce tasks are re-launched on other nodes, but not only.
Some map tasks have also to be run again.
Therefore, it slows down the application a lot more than if the node had crashed before the completion of map tasks.


\section{Application specific optimization in Hadoop}

\paragraph{Question 2.1}

During an execution, it is possible to choose when to start the reduce tasks compared to the completed map tasks.
At first, it may seem better to wait for all map tasks to end before running reduce tasks,
but in some situations starting reduce tasks earlier may lead to better performances.
This optimization is application dependent.
Starting reduce tasks earlier spreads out the data transfer over time, which is good if performances are limited by the network.
But on the other hand, reduce tasks occupy slots that may have been used by other tasks.

We used 5 nodes in \texttt{parapluie} cluster.
In this question, we generated with \texttt{randomwriter} a dataset of 5 Go, and with \texttt{randomtextwriter} a dataset of 1 Go.
We denote the reduce start paramater by $p$.
This mean that reduce tasks start when fraction $p$ of map tasks are completed.
We set $p \in \lbrace 0.05, 0.5, 1 \rbrace$.
We run sort benchmark on the 5 Go dataset, and the word count benchmark on the 1 Go dataset.

\begin{table}[!ht]
    \centering
\begin{tabular}{|c|c|c|c|}
    \hline
    \backslashbox{Benchmark}{Reduce slow-start} & 5\% & 50\% & 100\% \\
    \hline
                Sort             &   \figurename~\ref{2.1.sort.5}   &  -   & \figurename~\ref{2.1.sort.100} \\
    \hline
                Word count             &   \figurename~\ref{2.1.wc.5}   &  \figurename~\ref{2.1.wc.50}    & \figurename~\ref{2.1.wc.100} \\
    \hline
\end{tabular}
\end{table}

Sort and word count are applications with different behaviors.
Sort is more reduce-oriented.
That's why we can see that starting earlier reduce tasks improves significantly the execution runtime.
On the contrary, word count is map intensive, and the reduce slow-start value has little influence on the execution runtime.

\newpage
\appendix

%% To easily comment if compilation of all figures is
%% too long and is annoying when fixing small things
\section{Question 1.1}

\begin{figure}[!ht]
    \centering
    \includegraphics[width=\textwidth]{job_201703130805_0004_1489393243124_fldemoor_sorter_attempts.png}
    \caption{Question 1.1: Without speculation, without stress}
    \label{1.1.specOff.noStress}
\end{figure}
\newpage

\begin{figure}[!ht]
    \centering
    \includegraphics[width=\textwidth]{job_201703130805_0009_1489394222800_fldemoor_sorter_attempts.png}
    \caption{Question 1.1: With speculation, without stress}
    \label{1.1.specOn.noStress}
\end{figure}

\begin{figure}[!ht]
    \centering
    \includegraphics[width=\textwidth]{job_201703130805_0005_1489393310535_fldemoor_sorter_attempts.png}
    \caption{Question 1.1: Without speculation, with stress}
    \label{1.1.specOff.Stress}
\end{figure}
\newpage

\begin{figure}[!ht]
    \centering
    \includegraphics[width=\textwidth]{job_201703130805_0007_1489393851621_fldemoor_sorter_attempts.png}
    \caption{Question 1.1: With speculation, with stress}
    \label{1.1.specOn.Stress}
\end{figure}
\newpage

\newpage


\end{document}
